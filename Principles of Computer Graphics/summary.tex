% --------------------------------------------------------------
% This is all preamble stuff that you don't have to worry about.
% Head down to where it says "Start here"
% --------------------------------------------------------------
 
\documentclass[11pt]{article}
 
\usepackage[margin=0.8in]{geometry} 
\usepackage{amsmath,amssymb}
\usepackage{hyperref}
\usepackage{amsfonts}
\usepackage{graphicx}
\hypersetup{
  colorlinks   = true, %Colours links instead of ugly boxes
  urlcolor     = blue, %Colour for external hyperlinks
  linkcolor    = blue, %Colour of internal links
  citecolor   = red %Colour of citations
}

\usepackage{algorithm2e}


\usepackage{tcolorbox}

 
\newtheorem{theorem}{Satz}
\newtheorem{acknowledgement}[theorem]{Acknowledgement}
\newtheorem{axiom}[theorem]{Axiom}
\newtheorem{case}[theorem]{Case}
\newtheorem{claim}[theorem]{Claim}
\newtheorem{conclusion}[theorem]{Conclusion}
\newtheorem{condition}[theorem]{Condition}
\newtheorem{conjecture}[theorem]{Conjecture}
\newtheorem{corollary}[theorem]{Corollary}
\newtheorem{criterion}[theorem]{Criterion}
\newtheorem{definition}[theorem]{Definition}
\newtheorem{example}[theorem]{Example}
\newtheorem{exercise}[theorem]{Exercise}
\newtheorem{lemma}[theorem]{Lemma}
\newtheorem{notation}[theorem]{Notation}
\newtheorem{problem}[theorem]{Problem}
\newtheorem{proposition}[theorem]{Proposition}
\newtheorem{remark}[theorem]{Remark}
\newtheorem{solution}[theorem]{Solution}
\newtheorem{summary}[theorem]{Summary}
\newenvironment{proof}[1][Beweis]{\textbf{#1.} }{\ \rule{0.5em}{0.5em}}
\begin{document}
 
% --------------------------------------------------------------
%                         Start here
% --------------------------------------------------------------
 
\title{Principles of Computer Graphics - Summary}%replace X with the appropriate number
\author{Andy Tràn}
 
\maketitle %shows the current date of the Zusammenfassung

This will be a very personalised summary for me to use to study for the course Principles of Computer Graphics (CSCI3260). It might be complete, it might not be, it will probably not be. For questions you can refer to \href{mailto:andtran@ethz.ch}{andtran@ethz.ch}. This summary is based of the lecture notes and should be used as a supplement to the lectures. 

\tableofcontents

\newpage

\section{Lecture 1 - 06/09/2022: Introduction, Display and Colour}
\subsection{Important organizational stuff}
\begin{itemize}
    \item \href{mailto:pheng@cse.cuhk.edu.hk}{pheng@cse.cuhk.edu.hk}, office hours: Thursday 2:30 PM - 4:30 PM, SHB 929
    \item \textbf{Lecture hours:}  Tuesday 10:30 am - 12:15 pm, Thursday 11:30 am - 12:15 PM
    \item \textbf{Tutorial hours}: Monday 3:30 pm - 4:15 pm, Thursday 5:30 pm - 6:15 pm 
    \item \textbf{Reference book}: Fundamentals of Computer Graphics by Peter Shirley (not necessary), OpenGL Programming Guide
    \item \textbf{Course}: Consists of three parts: introduction, basics in graphics, more about graphics 
    \item \textbf{Grading}: (2) programming assignments 0.25, course project 0.20, mid-term exam 0.25, final exam 0.30
    \item \textbf{Release}: Assignment 1 - release 12/9, deadline 02/10, assignment 2 - release 3/10, deadline 30/10, course project - release 31/10, deadline 27/11, mid-term exam 18/10 10:30 - 12:15
\end{itemize}

\subsection{Display and Colour}
Lecture Outline
\begin{itemize}
    \item Display Devices and Basic Terminologies
    \item Frame Buffer (Memory to Display)
    \item Color Space: RGB, CMY, HSV, YIQ, CIE YZ
    \item ALpha Channel and Double Buffering
\end{itemize}

\subsubsection{Display Devices}
\textbf{Mechanism:}  shoot electrons with varying energy through vertical and horizontal deflectors to hit spot on screen, phosphors on screen jump to excited state when hit by electrons, emit monochromatic light when they drop to rest state
\newline
\textbf{Random scan/Vector scan:}  give instruction and follow instruction
\newline
\textbf{Raster scan:}  you go in a line and activate a line, and you turn each line on, where each spot on the screen is called a pixel. You shoot the gun, at the end of the line you turn it off and go back to the start, which is called \textbf{retrace}. There is a difference between horizontal and vertical retrace, horizontal is per line, vertical for each following line.
\newline
\textbf{Interlacing:} trick to get less flicker out of fixed signal bandwidth, it's like doubling the framerate, for example let's say we have 30Hz, we send two signals but each with a time difference between each other. For example to line 0 we send at 0 and 1, line 1 we send at 2 and 3. 
Result: doubling the perceived the framerate, without costing more bandwidth
\newline
Flat-Panel displays, two classes
\begin{itemize}
    \item \textbf{Nonemissive displays}: LCD (optical effects to split light)
    \item \textbf{Emissive display}: field emission display (FED), light-emitting diode (LED), organic light-emitting diode (OLED). Which is more power efficient
\end{itemize}
\noindent
On LCD: we block light instead of emitting the correct light.
\newline
FED: thousands of micro-electron guns

\subsubsection{3D Displays overview}
Two human visual cues used 
\begin{itemize}
    \item Stereopsis: seeing 2 slightly different images in each eye
    \item Motion Parallax: seeing slighty different images as you move around
\end{itemize}
\noindent
Terms used:
\begin{itemize}
    \item Stereoscopic: difference image to each eye, viewer must wear special glasses
    \item Autostereoscopic: different image to each eye, does not require special glasses
    \item Multi-view: different images depending on viewer's position
\end{itemize}
Note: Multi-view can be combined with used in combination of the others
\newline
\subsubsection{Stereoscopic and Autostereoscopic displays}
\begin{itemize}
    \item Stereoscopic displays (common): two approaches
     \begin{enumerate}
        \item using circularly polarized glasses (like in cinemas)
        \item using active shutter glasses (requires batteries, for example 3D TVs)
    \end{enumerate}
    \item Autostereoscopic displays: two approaches
    \begin{enumerate}
        \item Lenticular lens: bright but blurry, old
        \item Parallax barriers, darker but sharper, like Nintendo 3DS
    \end{enumerate}
\end{itemize}
\noindent
Downside of Autostereoscopic displays: usually limited to 1 or a very few viewers, and narrow sweet spot for viewing 3D.

\subsubsection{Multi-view displays}
Usually enabled by tracking the person's head.

\subsection{Frame Buffer}
Graphical storage (memory) and transformation hardware for digital images. We consider computer images as digital, we want to quantify a space into units (pixels).

\subsubsection{Greyscale/Monochrome Frame Buffer}
\begin{itemize}
    \item Intensity of the raster scan beam is modulated according to the contents of a frame buffer
    \item Each element of the frame buffer is associated with a single pixel on the screen
\end{itemize}
Each marker corresponds to a pixel on the computer screen, remember rasterizing from the beginning of this lecture

\noindent
Note: digital to analog converted (DAC)

\subsubsection{Resolution}
Determined by 
\begin{itemize}
    \item number of scan lines
    \item number of pixels per scan line
    \item number of bits per pixel
\end{itemize}

\subsubsection{Colours}

\begin{itemize} 
    \item 1 bit: B or W, 8 bit: 0 pure black to 255 pure white, with colours in between. To have true colour we need 8 bit per RGB
    \item To produce colours we mix intensities of each colour, to have a full spectrum we need a monitor which supports 256 voltages for each colour, the description of each colour in frame buffer memory is called \textbf{channel}. The term \textbf{truecolour (24 bits)} is for systems which the frame buffer stores the values for each channel (3 channels for RGB) 
    \item \textbf{Color table}: for few bits per pixel, we have to map non-displayable colours to displayable ones. We can remap color table entries in software. 
\end{itemize}

\subsubsection{Look Up Tables (LUT)}
Pseudo color: assign computed values systematically to a gray or color spectrum to indicate differences, for example height, speed, etc. 

\subsection{Color Space}
\begin{enumerate}
    \item RGB: additive color space, used for displays
    \item CMY: subtractive color space, used for printing
    \item HSV: (H circular, S distance from axis, V brightness), corresponds to artistic concepts of tint, shade, and tone
    \item YIQ: (Y luminance, I orange-cyan hue, Q green-magenta hue), exploits properties of the visual system, used in TV broadcasting
    \item XYZ system: defined in terms of three color matching functions  
\end{enumerate}

\begin{definition}[Gamut] 
    device's range of reproducible color
\end{definition}

\subsection{Alpha Channel and Double Buffering}
\subsubsection{Alpha Channel}
\textbf{Idea:}  we store one color per pixel, but we get hard edges. So we introduce an alpha channel next to the RGB channel, to blend with the lower layers to smoothen it out. Can be regarded as \textbf{1 - transparency} or \textbf{opacity}. 

\begin{example}[Blending]
    We have a source and destination image, we can overlay them and the alpha value denotes of how much percentage we see each image when we overlay them
\end{example}

\subsubsection{Image Matting} 
What part of the image we want to keep, using a mask.

\subsection{Double Buffering}
\begin{problem}
    what happens when we write to the frame buffer while it's being displayed?
\end{problem}
\begin{solution}[Double-buffering]
    \begin{enumerate}
        \item Render to the the back buffer and swap when rendering is done
        \item Double the memory
    \end{enumerate}
\end{solution}

\section{Lecture 2 - 08/09/2022: Useful 2D and 3D mathematics}
\subsection{Coordinate Systems}
\subsubsection{2D Cartesian Reference Frames}
There are two ways of using this system: (a) starting at the lower-left screen corner, (b) starting at the upper-left screen corner. 

\subsubsection{Polar Coordinates in the XY plane}
We start from a center, with a radial distance \textbf{r} and the angular displacement $\theta$   from the horizontal   
\newline
We can convert it to the cartesian system: $x = r cos \theta$ and $y = r sin \theta$ 
\newline
To polar system: $r = \sqrt{x^2 + y^2}$ and $ \theta = tan^{-1}(\frac{y}{x})$    
\newline
\textbf{Definition}: $\theta = \frac{s}{r}$, where $\theta$ is the angle subtended by the circular arc of length s and r
\newline
We know: $P = \frac{2\pi r}{r} = 2\pi$, total distance around P
\subsubsection{3D cartesian reference frames}
\textbf{Right-handed system}: Take your right hand, palm towards you, thumb (positive x direction) to the right, index finger up (positive y direction), middle finger towards you (positive z direction). Yes, the back of you is positive z.
\newline 
\textbf{Left-handed system}: Like RHS, but this time away from you is positive z. (Think of how to use your hand to show this)
\newline
In OpenGL: right handed (common), DirectX free to choose   
\subsubsection{Cylindrical-coordinate System}
\begin{enumerate}
    \item The surface of constant r is a vertical cylinder
    \item The surface of constant $\theta$ is a vertical plane containing the Z-axis
    \item The surface of constant z is a horizontal plane parallel to the Cartesian XY plane
    \item Transformation from a cylindrical coordinate specification to a cartestian reference system 
\end{enumerate}

$X = r cos \theta $, $Y = r sin \theta$, $ Z = z$

\subsubsection{Spherical-coordinate system}
Which is like polar coordinate in 3D space, we have $P(r,\theta,\phi)$
\newline
it holds that $x = r cos \theta sin \phi$, $y= r sin \theta sin \phi$, $z r cos \zeta$    

\begin{definition}[Angles in 3D] We define it as $ \omega = \frac{A}{r^2}$, total area is $\frac{4\pi r^2}{r^2} = 4\pi$ 

    
\end{definition}

\subsection{Points and Vectors}
\textbf{2D vector}:  $V = P_2 - P_1 = (V_x, V_y)$, length is defined as $\sqrt{V_x^2 + V_y^2}$, angle is $\alpha = tan^{-1}(\frac{V_y}{V_x})$ 
\newline
\textbf{3D vector}: V is same as 2D but with one more value, length is defined equivalently with $V_z$. We have three direction angles %todo

We have the following rules
\begin{enumerate}
    \item Addition: $V_1 + V_2 = (V_{1x} + V_{2x}, V_{1y} + V_{2y}, V_{1z} + V_{2z})$
    \item Scalar multiplication: $aV = (aV_x, aV_y, aV_z)$
    \item Scalar product: $V_1 \cdot V_2 = |V_1||V_2|\cos \theta$, from there you can derive $\theta$    
    \item Normalization: $\frac{V}{|V|}$, so its own product is 1
    \item Perpendicular: $|A||B|\cos \theta = A \cdot B =
     \begin{cases}
        0 & \text{if } \theta = 90 deg\\
        > 0 & \text{if }\theta < 90 deg\\
        < 0 & \text{otherwise}
    \end{cases}$  
    \item Cross product of two 3D vectors: $V_1 \times V_2 = u |V_1||V_2|\sin \theta$, where u is the unit vector that is perpendicular to both $V_1$ and $V_2$ %TODO\
    \item Basis vectors: we can specify the coordinate axes in any reference frame with a set of vectors, one for each axis
\end{enumerate}

\section{Lecture 3 - 13/09/2022}
\textit{Continuation of previous lecture}
Properties of cross product
\newline
\begin{itemize}
    \item Anti-commutative: $V_1 \times V_2 = -(V_2 \times V_1)$
    \item Non-associative: $V_1 \times (V_2 \times V_3) \neq (V_1 \times V_2) \times V_3$
    \item Distributive: $V_1 \times (V_2 + V_3) = (V_1 \times V_2) + (V_1 \times V_3)$   
\end{itemize}

\subsection{Useful 2D mathematics}
\begin{enumerate}
    \item Distance from P to line AB: \begin{itemize}
        \item Find line direction: $v = (x_2 - x_1, y_2 - y_1) = (d_x , d_y)$
        \item Find normal of line by swapping elements in v: $ n = (-d_y, d_x) = (y_1 - y_2, x_2 - x_1)$ 
        \item Normalize n and v as $\hat{n}, \hat{v}$  
        \item Use dot product to find h and l: $h = |(P-A) \cdot \hat{n}$, $l = (P - A) \cdot \hat{v}$  
        \item Need to check l against length of AB
        \item If $l < 0$ or $l > |AB|$, compute point-point distance  
    \end{itemize}
    \item Line-Line intersection: \begin{itemize}
        \item Express as parametric forms: $AB: (x_1, y_1) + t_{AB} (x_2 - x_1, y_2 - y_1)$ , $CD: (x_3, y_3) + t_{AB} (x_4 - x_3, y_4 - y_3)$ and set them both equal
        \item If no solution, that meas they are parallel
        \item Substitute back to the parametric form
    \end{itemize}
    \item Which-side test: given a point and a line, 
    Calculate %TODO
    \item Area of arbritrary polygons: polygon area $\frac{1}{2} \sum det ...$ %TODO
    \item Inside-outside test: \begin{itemize}
        \item Method 1: repeat the which-side test for each edge in order (only works for convex polygons)
        \item Method 2:Odd-intersection count, side =  $\begin{cases}
            \text{outside}, & number \equiv_2 0\\
            \text{inside}, & \text{otherwise}
        \end{cases}$
    \end{itemize}
    \item Linear Interpolation %TODO
    \item Barycentric coordinate: by means of area ratios %TODO
    \item Spherical linear interpolation %TODO
    \item Normal of a triangle %TODO
    \item Approximate normal at a vertex (vertex normal vs face normal): average the face normals of neighboring faces    
\end{enumerate}

\subsection{sth} %TODO
Properties
\begin{itemize}
    \item No fixed points under translation: all points move
    \item Multiple translations are order-independent, since addition is commutative 
\end{itemize}

\subsubsection{2D scaling}
Properties
\begin{itemize}
    \item Origin fixed: $x' = 0$ if $x=0$
    \item Order independent: $x'' = x' \cdot S_x' = x \cdot S_x \cdot S_x\  = x \cdot S_x' \cdot S_x$   
\end{itemize}
Single out \textit{arbitrary fixed point scaling} at $(x_0, y_0)$ as follows:
\begin{align*}
    x' &= x\cdot S_X + (1-S_x)x_0 \\
    y' &= y\cdot S_y  + (1-S_y)y_0
\end{align*} 

Rotate by $\theta$:
\begin{align*}
    x'&= r \cos (\theta + \phi) = r \cos \phi \cos \theta - r \sin \phi \sin \theta\\
    y' &= r \sin (\theta + \phi) = r \sin \phi \cos \theta + r \cos \phi \sin \theta
\end{align*} 
Multiple 2D rotations are order-independent
Fixed-point: origin dependent
\subsubsection{Other 2D transformations}
\begin{itemize}
    \item Reflection (about X/Y - axis): not equal to a rotation, except when reflecting over x and y, then it's a rotation
\end{itemize}
2D shearing: order dependent %TODO

\subsection{Matrix operations} %TODO need to check how to write matrixes quickly and easily slide 50

%TODO blablalba

\subsubsection{Homogenous representations}
\begin{enumerate}
    \item Translation cannot be represented using $2 x 2$  matrices and homogenous coordinates help
    \item We are left-multiplying the matrix on the vector/point 
\end{enumerate}



%FIX LECTURE 3
\section{REAL Lecture 3: Hierarchical Model}
\subsection*{Lecture Outline}
\begin{enumerate}
    \item Graphics Primitives: Points, Lines and Triangles
    \item Data structure: vertex list and index list
    \item Hierarchical Structure
    \item View-world or Modelview transformations
    \item Basic scenegraph concept
\end{enumerate}

\subsection{Graphics Primitives: Points, Lines and Triangles}

\begin{itemize}
    \item \textbf{Idea:} It's all about \textbf{coordinates} and \textbf{connectivity}
    \item \textbf{Meaning:} wireframe and meshes are built up by points, lines, or triangles. We call this \textbf{tesselation/triangulation}
    \item \textbf{Most efficient input type for rendering polygons:}  \textbf{Triangle strip} or \textbf{triangle fan}. The differences between the two is the following, triangle strip is a set of triangles which share vertices (can share different vertices), while triangle fan has one \textbf{(!)} shared center. 
\end{itemize}

%TODO: include img of TS and TF, they are in the ./img/folder   
\subsection{Data structure: vertex list and index list}
We have two kinds of data strucutres, namely vertex lists and index lists. 
\begin{itemize}
    \item \textbf{Vertex list:} you store all the vertex coordinates in one array \begin{itemize}
        \item Benefit: sequential memory access
        \item Negative: very likely to have duplicated vertices in the array, waste of storage
    \end{itemize}
    \item \textbf{Index list/Indexed Face Set:} we have two arrays, one called the \textbf{face list}. That one contains all the polygons and the corresponding vertices. While the \textbf{vertex list} contains the coordinates for each vertex. \begin{itemize}
        \item Benefit: reuse vertices and keeps a compact vertex list
        \item Bad: random memory accesses which lead to cache misses
    \end{itemize}
\end{itemize}

\textit{Note:} (1) triangle strip can be implemented on either data structure, (2) in some APIs these data structures are built in, (3) indexing can start with 0 or 1.

\subsection{Hierarchical structure}
A scene is composed of the multiple objects, which are composed of subjects as well. We can create a hierarchical model and break it down. Each objects has its own polygon list and associated vertex list.
\subsection{View-world or Modelview transformations}
\begin{itemize}
    \item \textbf{Idea:}  we model objects in a model space, which is pretty dumbed down in a cartesian matter where we start from the origin. (\textbf{Model space}) This usually doesn't reflect the reality and we want to model it to the common world (\textbf{World space}) and then to our eyes perspective (\textbf{Eye perspective}). 
    \item \textbf{Example:}  we create a moon in the object space, we put it in the sky, and then we look at it. 
    \item \textbf{Concrete:} An object is defined as $P_{obj} = (x,y,z)^T$ to get it into the world coordinates we need to transform it, namely $M_{obj2world} \times P_obj$ and to get the eye coordinates, $M_{world2eye}\times M_{obj2world} \times P_obj$.
    \textbf{Note:} Since we usually don't care about the steps inbetween, we usually merge $M_{obj2world}$ and $M_{world2eye}$ into one matrix called the \textbf{modelview/viewworld} $M_{modelview}$.
\end{itemize}


Notes about the modelview Matrix $M_{modelview}$:
\begin{itemize}
    \item OpenGL puts the eye-point at the origin looking towards the negative z-axis
    \item OpenGL is a state machine: it has an internal memory storage for the modelview matrix
    \item When calling transformation operations, the kernel constructs a matrix for the transform and right multiplies it with its internal modelview matrix.
\end{itemize} %TODO: explain what point 3 does exactly?

%TODO: show to illustration from the slides, we have mv_steps.png to import
 
\subsection{Basic scenegraph concept}
\begin{itemize}
    \item Organize the whole model hiearchy as a tree structure
    \item Examples of language who do that: VRML, OpenSG
\end{itemize}

How does this work exactly? We have group nodes, who associate nodes into hierarchies, leaf nodes, who contain all the descriptive data of objects in the virtual world used to render them.

\subsubsection*{Avoid Modeling Glitches}
\begin{enumerate}
    \item Avoid T-Join \begin{itemize}
        \item Problem: we have edges which sometimes don't touch due to computation and rounding errors of floats
        \item Solution: we break a triangle into two triangles.
    \end{itemize}
    \item Avoid overlapping polygons in your model \begin{itemize}
        \item Problem: overlapping triangles give flipping colors (\textbf{z fighting})
        \item Solution: create another triangle for the overlapping area, or turn off depth test when drawing the next triangle
    \end{itemize}
    
    
\end{enumerate}
 

\section{Lecture 4: Interactive 3D Control}
\subsection*{Lecture Outline}
\begin{enumerate}
    \item Interactive Control Concept
    \item Interactive Translation relative to screen
    \item Interactive rotation: rolling ball
\end{enumerate}

\subsection{Interactive Control Concept}
\begin{itemize}
    \item \textbf{Idea:} we use the mouse as an input device for 3D viewing control
    \item \textbf{Approach:} by modifying the (base) 4x4 Modelview matrix incrementally accordingly to the mouse motion, we can control our 3D viewing interactively. 
\end{itemize}

\textbf{Normal Modelview Transformation Equation:}  assume the modelview matrix is affine. In particular, it has translation, rotation and scaling only. If so we can write the Modelview transformation like this:

\[
    \begin{bmatrix} x_{eye}\\ y_{eye} \\ z_{eye} \\ 1 \end{bmatrix} = \begin{bmatrix} m_{11} & m_{12} & m_{13} & T_x \\ 
        m_{21} & m_{22} & m_{23} & T_y \\
        m_{31} & m_{32} & m_{33} & T_y \\
        0 & 0 & 0 & 1\end{bmatrix} \begin{bmatrix} x_{world} \\ y_{world} \\ z_{world} \\ 1 \end{bmatrix} 
.\]


\subsection{Interactive Translation relative to screen}
Left-Multiply a translation matrix to the existing modelview. We can have a screen-aligned translation, since we don't need to use the fixed point rule for translations.

%TODO: show OpenGL implementation, all in slides

\subsection{Interactive rotation: rolling ball}
If we want to rotate our existing modelview, one of the most common mistakes being made is left-multiplying the rotation matrix to our modelview. We need to apply the \textbf{fixed point rule}
\begin{enumerate}
    \item Undo the translation $(T_x, T_y, T_z)$ in the modelview matrix
    \item Left-multiply the rotation matrix
    \item Redo the translation $(T_x, T_y, T_z)$ 
\end{enumerate}
\noindent
Now we need to construct the rotation matrix, to map our 2D mouse motion to a 3D rotation.
\begin{itemize}
    \item \textbf{Method:}  rolling ball
    \item \textbf{Idea:} imagine the object is inside a glass ball of radius r.
    \item To construct the rotation matrix: \begin{enumerate}
        \item Put the mouse motion in an XY-eye space: (dx, dy, 0)
        \item Axis of rotation perpendicular to the motion vector: (-dy, dx, 0)
        \item Angle of rotation relative to motion vector length: $\sqrt{dx^2 + dy^2}$ 
    \end{enumerate}

    %TODO: include graphic of OGL implementation
    
\end{itemize}

\section{Lecture 5: Cameras, Projections, and Clipping}
\subsection*{Lecture Outline}
\begin{enumerate}
    \item Pin-hole camera
    \item Family of Projections
    \item Parallel Projection
    \item Perspective Projection
    \item OpenGL Projection Model: 4x4 projection matrix
    \item Clipping and Perspective Division
\end{enumerate}




\end{document}