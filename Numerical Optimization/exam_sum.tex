% Basic stuff
\documentclass[a4paper]{article}
\usepackage[10pt]{extsizes}

% 3 column landscape layout with fewer margins
\usepackage[landscape, left=0.75cm, top=1cm, right=0.75cm, bottom=1.5cm, footskip=15pt]{geometry}
\usepackage{flowfram}
\ffvadjustfalse
\setlength{\columnsep}{1cm}
\Ncolumn[<9]{3}
\onecolumn[9]

% define nice looking boxes
\usepackage[many]{tcolorbox}

% a base set, that is then customised
\tcbset {
	base/.style={
		boxrule=0mm,
		leftrule=1mm,
		left=1.75mm,
		arc=0mm, 
		fonttitle=\bfseries, 
		colbacktitle=black!10!white, 
		coltitle=black, 
		toptitle=0mm, 
		bottomtitle=0mm,
        bottom=0mm,
        top=0mm,
		title={#1}
	}
}

\definecolor{brandblue}{rgb}{0.34, 0.7, 1}
\newtcolorbox{mainbox}[1]{
	colframe=brandblue, 
	base={#1}
}

\newtcolorbox{subbox}[1]{
	colframe=black!20!white,
	base={#1}
}

% Mathematical typesetting & symbols
\usepackage{amsthm, mathtools, amssymb} 
\usepackage{marvosym, wasysym}
\allowdisplaybreaks

% Tables
\usepackage{tabularx, multirow}
\usepackage{makecell}
\usepackage{booktabs}
\renewcommand*{\arraystretch}{2}

% Make enumerations more compact
\usepackage{enumitem}
\setitemize{itemsep=0pt}
\setenumerate{itemsep=0pt}

% To include sketches & PDFs
\usepackage{graphicx}

% For hyperlinks
\usepackage{hyperref}
\hypersetup{
	colorlinks=true
}

\usepackage{titlesec}
\titlespacing*{\section}
{0pt}{0ex}{1ex}
\titlespacing*{\subsection}
{0pt}{0ex}{1ex}


% section smaller
\usepackage{sectsty}
\sectionfont{\large}
\subsectionfont{\small}


\begin{document}
\section{Linear Algebra}
\begin{itemize}
    \item Linear Indepent: m vectors are \textbf{l.i.} if $c_1 \mathbf{x_1} + ... + c_m\mathbf{x_m} = \mathbf{0}$, only when $c_1 = ... = c_d = \mathbf{0}$ 
    \item Standard basis of $\mathbb{R}^d$ is composed of $e_1, ..., e_d$    
    \item Euclidean length: $\sqrt{x_1^2 + \dots + x_d^2}$ 
\end{itemize}


\begin{mainbox}{Norm function}
    A function $\left\lVert \cdot \right\rVert: \mathbb{R}^d \to \mathbb{R}$ 
    is called a norm if it satisfies \begin{enumerate}
        \item $\left\lVert x\right\rVert \geq 0 \forall \mathbf{x}\in \mathbb{R}^d$ and for $\mathbf{x}\neq \mathbf{0}$ we have $\left\lVert x\right\rVert > 0$ 
        \item $\forall c \in \mathbb{R}, $ $|c|\left\lVert x\right\rVert $  
        \item $\forall \mathbf{x,y}\in \mathbb{R}^d, \left\lVert x+y\right\rVert \leq \left\lVert x\right\rVert + \left\lVert y\right\rVert $   
    \end{enumerate}
\end{mainbox}

\begin{subbox}{$l_p-$norm function}
    The norm is defined as $\left\lVert x\right\rVert_p := (|x_1|^p + \dots + |x_d|^p)^\frac{1}{p}$ 
\end{subbox}
\textbf{Note:} $\left\lVert x\right\rVert_p = \lim_{p\to \infty}\left\lVert x\right\rVert_p = \underset{1\leq i \leq d}{max} |x_i| $ 
\newline
Also, $l_p-norms$ are decreasing in $p\geq 1$, namely $1 \leq p \leq q \leq \infty \Rightarrow \left\lVert x\right\rVert_p \geq \left\lVert x\right\rVert_q$.
\newline
The inner product $\langle x,y \rangle = x^Ty$ is positive-definite for all $\langle x,x\rangle$ , symmetric for all $\mathbf{x,y}$ , and linear for all $\mathbf{x_1, x_2, y}, c\in\mathbb{R} $ . 
\newline
\textbf{Cauchy-Schwartz inequality: } $|\langle x,y\rangle| \leq \left\lVert x\right\rVert \left\lVert y\right\rVert$. As a result we have $\cos \theta = \frac{\langle x,y\rangle}{\left\lVert x\right\rVert\left\lVert y\right\rVert  }$   

\begin{mainbox}{Mutual Orthogonal}
    A set of vectors is called \textbf{mutually orthogonal} if $\forall i \neq j : <x^{(i)}, x^{(j)}> = 0$ and a set of mutually orthogonal vectors is linearly independent.
\end{mainbox}
\textbf{Note:} we call a set orthonormal if $<x^{(i)}, x^{(j)}> \delta_i,j$  

Matrix notation:
$A= \left( \begin{smallmatrix} a_{1,1} & a_{1,2} & \dots & a_{1,n} \\
a_{2,1} & a_{2,2} & \dots & a_{2,n} \\
. & . & \dots & .\\
a_{m,1} & a_{m,2} & \dots & a_{m,n} \end{smallmatrix} \right)$ 

The product is defined as $\left[AB\right]_{i,j} = \sum_{k=1}^{n} a_{i,k}b_{k,j}$ 

\begin{itemize}
    \item Range of matrix, the subspace of all vectors following from linear combinations of A's columns: $\mathcal{R}(A) := \{Ax : x\in\mathbb{R}^n \}$
    \item Rank of matrix: dimension of subspace $\mathcal{R}(A)$
    \item $Rank(A_{m\times n}) \leq min(m,n)$ and full-rank if $Rank(A) = min(m,n)$
    \item Null space of a matrix, the subspace of all vectors which A maps to $\mathbf{0}$: $\mathcal{N}(A) := \{x: Ax =\mathbf{0}\}$
    \item Determinant: $det(A) = \sum_{j=1}^{n} (-1)^{i+j}a_{i,j}det(A_{(i,j)})$     
\end{itemize}

Equivalent sayings:
\begin{itemize}
    \item A is invertible
    \item A is non-singular, $det(A) \neq 0$ 
    \item A is full rank, $Rank(A) = n$
    \item A has linearly independent rows or columns
    \item A has a zero null space $\mathcal{N}(A) = 0$
    \item A has full range $\mathcal{R}(A) = 0$   
\end{itemize}

\begin{subbox}{Eigenvectors and Eigenvalues}
    For a vector $\mathbf{v} \neq \mathbf{0}$, it's an eigenvector for its eigenvalue $\lambda$ such that: $Av = \lambda v$    
    
\end{subbox}
Can be calculated by solving $det(A - \lambda I_n)$  

\begin{mainbox}{Spectral theorem}
    For a symmetric matrix A there exists a spectral decomposition. Such that \[
    A = \sum_{i=1}^{n} \lambda_i v_i v_i^T = V \Lambda V^T
    .\],
    where $ V= [v_1, ..., V_n]$ and $\Lambda = diag(\lambda_1, \dots, \lambda_n)$ such that $\lambda_1 \geq \dots \geq \lambda_n$ and $v_i$ is normalized 
\end{mainbox}

A symmetric matrix is called PSD if for every vecotr $x\in\mathbb{R}^n$ we have $x^TAx \geq 0$. Strictly definite if it holds strictly for $\mathbf{x} \neq \mathbf{0}$   

Theorem:
A is psd $\Leftrightarrow $ all its eigenvalues are non-negative $\Leftrightarrow$ we have a matrix H such that $A=HH^T$  

\begin{subbox}{Partial Order for Matrices}
    \begin{itemize}
        \item $A \succeq B$ if $A-B$ is PSD
        \item $A\succ B$ if $A-B$ is positive definite (PD)
        \item $A\preceq B$ if $A-B$ is negative semidefinite
        \item $A \prec B$ if $A-B$ is negative definite      
    \end{itemize}
    $A\nsucceq  B$ does not imply $A \preceq B$ 
    
\end{subbox}




\section{(Multivariable) Calculus Recap}
\section*{Lecture 5}
\begin{subbox}{Special sets}
    Epigraph:
$
    epi f := \{(\mathbf{x},t): \mathbf{x} \in \mathbb{R}^n, t \geq f(\mathbf{x})\}
     $\newline
    Contour set:
    $
    C_f(t) := \{\mathbf{x}\in \mathbb{R}^n: f(\mathbf{x}) = t\}
      $\newline
    Hyperplane:
    $
    H = \{\mathbf{x} \in \mathbb{R}^n: \mathbf{a}^Tx = b \} 
    $\newline
    Halfspace (change $\leq$ to $\geq$ for +  )
    $H_- = \{\mathbf{x} \in \mathbb{R}^n: \mathbf{a}^Tx \leq b \} 
    $\newline


\end{subbox}

A function is linear if $\forall x,y\in\mathbb{R}^n, c\in \mathbb{R} f(cx + y) = cf(x) + f(y)$ 

\begin{subbox}{General Quadratic Form}
    We can write a quadratic function, in terms of $A\in \mathbb{R}^{n\times n}, b\in \mathbb{R}^n, c\in \mathbb{R}$:
    \[
    f(\mathbf{x}) = \frac{1}{2}\mathbf{x}^TA\mathbf{x} + \mathbf{b}^T\mathbf{x} + c
    .\] 
\end{subbox}

Common gradients and hessians:
\begin{mainbox}{Gradient and Hessians}
    A function $f: \mathbb{R}^n \to \mathbb{R}$ has a gradient defined as $\nabla f(x) = \left[\frac{\delta f(x)}{\delta x_1}, \dots, \frac{\delta f(x)}{\delta x_n}\right]^T$ and hessian is the second derivative (square matrix.)   
\end{mainbox}
\textbf{Math Rules:} 
\begin{itemize}
    \item Derivative for $f:\mathbb{R}^n \to \mathbb{R}^m$ :
    $ df(x) = \left[\begin{smallmatrix} 
        \frac{df_1}{dx_1} & ... & \frac{df_1}{dx_n}\\
        \frac{df_2}{dx_1} & ... & \frac{df_2}{dx_n}\\
        ... & ... & ... \\
        \frac{df_m}{dx_1} & ... & \frac{df_m}{dx_n}
    \end{smallmatrix}\right] \in \mathbb{R}^{m\times n}$ 
    \item Gradient: we can calculate gradient for $f: \mathbb{R}^n \to \mathbb{R}$. It's defined as $\nabla f(x) = df(x)^T$ 
    \item Chain rule: $d(g\circ f)(x) = dg(f(x)) \cdot df(x)$ 
    \item Product rule: $g(x)h(x) = dg(x)h(x) + g(x)dh(x)$ 
    \item $f(x)= \left\lVert Ax-b\right\rVert ^2, \nabla f(x) = 2A^T(Ax-b), Hess_f(x) = 2A^TA$
    \item $f(x) = \left\lVert Ax-b\right\rVert, \frac{A^T(Ax-b)}{\left\lVert Ax-b\right\rVert}, Hess_f(x)= \frac{A^TA}{\left\lVert Ax-b\right\rVert }  - \frac{(A^TAx-b)((x^TA^T-b^T)A)}{\left\lVert Ax-b\right\rVert^3 }$  
    \item $d\mathbf{0} = 0, d(\alpha \mathbf{X}) = \alpha d\mathbf{X}, d(\mathbf{X^{-1}}) = -\mathbf{X^{-1}}(\mathbf{dX})X^{-1}, d\mathbf{X}^T = (d\mathbf{X})^T, \frac{dx^Ta}{dx} = \frac{da^Tx}{dx} a, \frac{dx^TAx}{dx} (A+A^T)x, \frac{d}{ds}(x-As)^TW(x-As) = -2A^TW(x-As), \frac{d}{dx}(x-As)^TW(x-As) = 2W(x-As)$    
\end{itemize}



We can approximate functions using gradients and hessians:
$
f(x) \approx f(\mathbf{x_0}) + \nabla f(\mathbf{x_0})^T (\mathbf{x-x_0}) +
\newline
 \frac{1}{2}(\mathbf{x-x_0})^T H_f(\mathbf{x_0})(\mathbf{x-x_0})
$

\begin{mainbox}{Affine Functions}
    Let $f\in\mathbb{R}^d \to \mathbb{R}$ be a multivariable function. Then f is an affine function iff:
    $
    \forall x,y\in \mathbb{R}^d, \theta \in [0,1]: f(\theta x + (1-\theta)y) = \theta f(x) + (1-\theta)f(y)
    $
\end{mainbox}
\textbf{Convex function (one dimensional):} if we can replace $=$ above with $\leq$. A function f is \textbf{concave}  if $-f(x)$ is convex.
\newline
\textbf{Convex (multivariable):} $||\mathbf{x}||$, $||\mathbf{x}||^p, p\geq1$, $\log(\sum_i^d e^{x_i})$. \textbf{Concave:} $\sum_i^d x_i log(\frac{1}{x_i}), \newline
dom(f)=R^d_+$, $(x_1 x_2 \dots x_d)^\frac{1}{d}, dom(f)= \mathbb{R}^d_+$      
\newline
\textbf{Proposition:} A \textbf{set S is convex} iff $\forall x,y\in S, \theta\in[0,1]: \theta x + (1-\theta)y \in S$
\newline
\textbf{Examples of convex functions: } $x^p, p \geq 1$ or $p \leq 0$, $|x|^p, p\geq 1$, $e^{ax+b}$. \textbf{Concave:} $x^p, x\in \mathbb{R}_+, 0\leq p \leq 1$        
\begin{subbox}{Sublevel set}
    Let $S_f(t) = \{\mathbf{x} \in dom(f): f(\mathbf{x}) \leq t$. If f is convex, then $S_f(t)$ is a convex set for every t.  
\end{subbox}

\textbf{Examples}  of convex sets: hyperplanes, halfspaces, norm balls $(\{ \mathbf{x}: \left\lVert x \right\rVert \leq \epsilon )$ %TODO: maybe ill define it later before.
\newline
\textbf{Convexity preserving operations:} intersection, affine and inverse-affine mappings, linear fractional functions 
\newline
How to proof a function is convex:
\begin{enumerate}
    \item Verify the inequality
    \item Proof over epigraph and sub-level sets
    \item Gradients and hessians
    \item Convexity preserving operations
\end{enumerate}
\textbf{Convexity preserving operations:} positive scalar multiplication, addition of two convex functions, composition with affine functions $f(\mathbf{Ax+b})$, pointwise maximum $max\{f_1(x), \dots, f_k(\mathbf{x})\}$ if each $f_i$ is convex, composition $f(g(x))$, where f,g both convex and $f$ non-decreasing in every entry.

\begin{mainbox}{First-order convexity condition}
    A differentiable function f is convex iff its domain is convex and $\forall x,y\in dom(f): f(y) \geq f(x) + \nabla f(x)^T (y-x)$ 
\end{mainbox}
\begin{subbox}{Second-order convexity}
    Same as above, but this $\forall \mathbf{x}\in dom(f) : H_f(\mathbf{x}) \succeq 0$ (so PSD)  
\end{subbox}


\section{Optimization Problems}
Let the optimization problem be formulated as $\underset{x\in\mathbb{R}^d}{min}f(x)$ subject to $g_i(x) \leq 0$ for all $i$   
\begin{itemize}
    \item NP-Hard problems: \textit{proven}  to be intractable
    \item  Linear Programming Problems: if a problem can be rewritten as $\underset{x\in\mathbb{R}^d}{min} \textbf{c} ^T\textbf{x} $ subject to $A\textbf{x}  \leq \textbf{b} $ and $\textbf{x} \geq\textbf{0} $, with $f$ and $g_i \forall i$ affine.   If f is non-Affine, then it's a non linear programming task
    \item Convex optimization problem: if $f, g_i$ are all convex and we can rewrite it as $\underset{x\in\mathbb{R}^d}{\min}f(x) $ subject to $g_(x) \leq 0$ for all $i$ or $Ax=b$, else non-convex  
\end{itemize}

\begin{mainbox}{How to}
    \begin{itemize}
        \item Define optimization variables, i.e. $x\in\mathbb{R}^d$
        \item Define objective function 
        \item Define feasible set, or constraint functions, also must $x_i \geq 0$ for example? 
    \end{itemize}
    
\end{mainbox}

Examples:
\begin{itemize}
    \item LP: Transport task, manufacturing task, sorting task
    \item Convex problems: LP-problems, projection problem, distance computation problem, ridge regression
\end{itemize}

\textbf{Definition:} two problems are called \textit{equivalent} if their optimal solutions are in one-to-one correspondence.  

Feasible set:
$S=\{x\in\mathbb{R}^d: g_i(x) \leq \text{ for all } 1\leq i \leq m\}$ and $x^*$ is called the globally optimal solution if it minimized the object function. We also have a locally optimal solution. In convex problems they are the same.



\section{Extra}
\begin{mainbox}{Gram-Schmidt: orthogonalize a basis}
 \begin{enumerate}
    \item Let $proj_u(v) = \frac{\langle u, v \rangle}{\langle u, u\rangle}u$ 
    \item $u_1 = v_1 \Rightarrow e_1 = \frac{u_1}{\left\lVert u_1\right\rVert } $
    \item $u_2 = v_2 - proj_{u_1}(v_2) \Rightarrow e_2 = \frac{u_2}{\left\lVert u_2\right\rVert }$
    \item \dots
    \item $u_k = v_k - \sum_{j=1}^{k-1}proj_{u_j}(v_k) \Rightarrow e_k = \frac{u_k}{\left\lVert u_k\right\rVert }$ 
 \end{enumerate}

 We do this for all the k vectors we have and normalize them all
 
\end{mainbox}

\begin{mainbox}{Find eigenvectors}
    \begin{enumerate}[leftmargin=7pt]
        \item Solve for $det(A-\lambda I) \overset{!}{=} 0 $
        \item Set $\lambda = \lambda_i \forall i$
        \item Perform gauss elimination, and then get the systems of equations wrt one variable and set the variable to 1
        \item If necessary, normalize
    \end{enumerate}
\end{mainbox}

\begin{subbox}{Prove PSD}
    One of the easier ways to prove a matrix is PSD, is by $x^TAx$ and usually showing it's a norm squared. Another way is Sylvesters' criterion:
    Let $A^{(1)} = [a_{11}], A^{(n)} = A$, like $A^{(k)}$ be the $k\times k$ submatrix from topleft. Then $\Delta_k = det(A^{(k)})$    
    \begin{itemize}
        \item A spd $\Leftrightarrow$ $\Delta_i > 0, ..., \Delta_n > 0$
        \item A snd $\Leftrightarrow$ $(-1)^1 \Delta_1 > 0, ..., (-1)^n \Delta_n > 0$  
    \end{itemize}
      This is not always conclusive!
\end{subbox}
\end{document}